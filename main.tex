\documentclass[11pt,english]{article}
\usepackage[utf8]{inputenc}
\usepackage[T1]{fontenc}
\usepackage{babel}
\usepackage{biblatex}
\addbibresource{bibliography.bib}

% correct bad hyphenation here
\hyphenation{op-tical net-works semi-conduc-tor}
\begin{document}
\title{Use of Model-Driven Development to check for violations of the GDPR}
\author{
  Tijana Lalošević\\
  \texttt{tijana.vdn@gmail.com}
  \and
  Gordana Milosavljević\\
  \texttt{grist@uns.ac.rs}
  \and
  Goran Sladić\\
  \texttt{sladicg@uns.ac.rs }
  \\Faculty of Technical Sciences,\\ University in Novi Sad, Serbia
}


\date{}
\maketitle


\begin{abstract}
The General Data Protection Regulation (GDPR) is a law regulation issued by the European Union. The GDPR was adopted on 14 April 2016 and became enforceable beginning 25 May 2018. This regulation aims to give individuals control over their personal data across the European Union and unify regulations within the EU to facilitate international business. This regulation is a mixed blessing, on the one hand, it has brought many benefits to individuals, but on the other hand, it has provided many challenges and obstacles to companies and organizations that control or process personal data. Despite that the GDPR provides a unified and centralized law regulation, there is no universal software solution for checking GDPR compliance. Therefore, to overcome some of the security challenges, companies pay expensive manual checking executed by lawyers or create their own solutions that do not apply to other organizations and industries. In this paper, we propose a solution in the form of a privacy meta-model and OCL validations which provides automatical checking for violations of the first fifty arts of the GDPR. Also, we will present usage of this meta-model on an example of a bank and plan of future development.
\end{abstract}

\section{Introduction}

The world is an ever-changing paradigm, as each age has its crucial resource, as "the gold" of the 21st century, the data stood out. As such, they caused more and more frequent abuses, and their use had to be legally regulated. Digitalization and globalization have enabled the rapid growth and development of international business, but this has led to the problem of inconsistencies in legislation between the countries in which it operates. To overcome this problem, the European Union adopted the GDPR, which unified privacy management and data protection. On the business side, this regulation caused a great deal of difficulty as companies had to adjust their business to the new rules in a very short period. A large number of surveys have been conducted to determine how ready the market is for the new regulations. Based on Isaca survey \cite{isaca} the top five advantages related to GDPR compliance are:
\begin{itemize}
  \item Data discovery and mapping (59 per cent)
  \item Prioritizing GDPR compliance among other business priorities (47 per cent)
  \item Organizational education and change programs (45 per cent)
  \item Ensuring cross-departmental collaboration and buy-in (42 per cent)
  \item Preparation for data subject access or deletion requests (37 per cent).
\end{itemize}
Also, the Lipswitch survey\cite{lipswitch} has given similar results. This survey reveals that 52 per cent of the examined companies admitted they were not ready to apply the GDPR, also 35 per cent confessed to not knowing whether their IT policies and process were up to the job. The primary cause for this, to date, is that there is no single comprehensive solution that would be applicable and easy to integrate into all technologies. Therefore companies are most often forced to pay legal experts to study their business ecosystem, and based on that, the IT team designs a unique solution for their information system. The whole process is slow, expensive, and errors prone.
In this paper, we present a privacy meta-model with OCL validations\cite{ocl} based on a Model-Driven Development solution, which presents the first step towards an automatic and universal check for violations of the GDPR.

\section{Background}

\printbibliography
\end{document}
