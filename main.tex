\documentclass[11pt,english]{article}
\usepackage[utf8]{inputenc}
\usepackage[T1]{fontenc}
\usepackage{babel}
\usepackage{biblatex}
\addbibresource{bibliography.bib}

% correct bad hyphenation here
\hyphenation{op-tical net-works semi-conduc-tor}
\begin{document}
\title{Use of Model-Driven Development to check for violations of the GDPR}
\author{
  Tijana Lalošević\\
  \texttt{tijana.vdn@gmail.com}
  \and
  Gordana Milosavljević\\
  \texttt{grist@uns.ac.rs}
  \and
  Goran Sladić\\
  \texttt{sladicg@uns.ac.rs }
  \\Faculty of Technical Sciences,\\ University in Novi Sad, Serbia
}


\date{}
\maketitle


\begin{abstract}
The General Data Protection Regulation (GDPR) is a legal regulation on the use, protection and privacy of data issued by the European Union. The GDPR was adopted on 14 April 2016 and became enforceable beginning 25 May 2018. \cite{gdpr} This regulation aims to give individuals control over their data across the European Union and unify laws within the European Union to facilitate international business. This regulation is a mixed blessing. On the one hand, it has brought many benefits to individuals, but also, it has provided many challenges and obstacles to companies and organizations that control or process personal data. Despite unification and centralization of law regulation, there is no universal software solution for checking GDPR compliance. Therefore, to overcome some of the security challenges, companies pay expensive manual checking executed by lawyers or create solutions that do not apply to other organizations and industries. In this paper, we propose a solution in the form of a privacy meta-model, as also, OCL validations \cite{ocl} which provides automatical checking for violations of the first fifty arts of the GDPR. Besides, we will present usage of this meta-model on an example of a bank and plan of future development.
\end{abstract}

\section{Introduction}

\quad The development of technology and the discovery of new technologies have brought extraordinary changes in many living fields and the growth of living standards. These innovations are creating new goods that are becoming a source of demand for more and more people. Thus, the data stood out as the "gold" of the 21st century. As such, they caused more and more frequent abuses. One of the many problems is leakage of sensitive information, their sale on the black market, modifications, intellectual property thefts, and many others. Because of that reason, their use had to be legally regulated. Also, there was a need to introduce information security management. Information security risk management (ISRM) is the primary means by which organizations preserve the confidentiality, integrity and availability of information resources. \cite{WEBB20141} Digitalization and globalization have erased international borders and enabled the rapid growth and development of international business. Increased business movements have led to the problem of legislative inconsistencies between the countries in which it operates. This gap between regulations has made a place for many data thefts, resale and other abuses. To overcome this problem, the European Union adopted a legal framework, the GDPR, which unified privacy management and data protection. On the business side, this regulation caused a great deal of difficulty as companies had to adjust their business to the new rules in a brief period.
\newline \quad Many surveys conducted to determine how ready the market is for the new regulations. Based on Isaca survey \cite{isaca} the top five advantages related to GDPR compliance are:
\begin{itemize}
  \item Data discovery and mapping (59 per cent)
  \item Prioritizing GDPR compliance among other business priorities (47 per cent)
  \item Organizational education and change programs (45 per cent)
  \item Ensuring cross-departmental collaboration and buy-in (42 per cent)
  \item Preparation for data subject access or deletion requests (37 per cent).
\end{itemize}
\quad The Lipswitch survey \cite{lipswitch} has given the following results similar to the first survey. This survey reveals that 52 per cent of the examined companies admitted they were not ready to apply the GDPR, besides 35 per cent confessed to not knowing whether their IT policies and process were up to the job. These results are not unexpected, as it is the main cause for this is that there is a lack of a single comprehensive solution that would be applicable and easy to integrate into all technologies. The secondary cause was a misunderstanding of terms and expressions which have come with the GDPR. Therefore companies are most often forced to pay legal experts to study their business ecosystem. After that, based on those studies, the IT team designs a unique solution for their information system. The whole process is slow, expensive, and errors prone. \newline This paper aims to define a universal meta-model that is close to natural language that can describe the real scenarios of using personal data in the information system and check whether their usage follows the GDPR. To achieve this goal, we used Model-Driven Engineering (MDE) technologies \cite{mde}, such as UML \cite{uml} and Object Restricted Language (OCL). After that, we will analyze the expressiveness of the meta-model and the usability of OCL validations on the example of the bank case study. Finally, as it is the first step towards easy integration with software solutions and automatical verifying compliance of business processes to the GDPR, we will introduce our strategy for future development. \newline The GDPR is considered the toughest and rigorous privacy and security law in the world.  Its complexity is reflected in the fact that it includes all manipulations over the data. Starting with data collection, storage and transformation, transport, processing and ends with stop processing. It places special emphasis on the fact that the data owner and his role in data processing. The data owner is the only one who can manage the data, introducing terms such as complaint, withdrawal of consent and request for the erasure. We can see how comprehensive the GDPR is in that it covers all previously adopted regulations and safety standards. So in the overview of the works, we will also consider solutions that rely on previously adopted regulations. \newline The structure of the paper is as follows. In the second section, 'Related Work', we provide the related work. In the third section, ‘Privacy meta-model’, we present our meta-model with accompanying OCL validations. After that, in section ‘Case study’, we prove the use of the meta-model on the example of a bank case study. In section ‘Discussion’,  we discuss the results and limitations of the study. Finally, in section ‘Conclusion’, the paper has been concluded, and some future development has proposed.
\section{Related work}
Before the appearance of the GDPR, many different regulations and laws were in force. As we already mentioned before, the GDPR covers all of these regulations.  There have been many proposed solutions that deal with compliance of business processes with applicable regulations.
\section{Privacy meta-model}
\section{Case study}
\section{Discussion}
\section{Conclusion}
\printbibliography
\end{document}
