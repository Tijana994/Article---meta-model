\documentclass[11pt,english]{article}
\usepackage[utf8]{inputenc}
\usepackage[T1]{fontenc}
\usepackage{babel}

% correct bad hyphenation here
\hyphenation{op-tical net-works semi-conduc-tor}
\begin{document}
\title{Use of Model Driven Development to check for violations of the GDPR}
\author{
  Tijana Lalošević\\
  \texttt{tijana.vdn@gmail.com}
  \and
  Gordana Milosavljević\\
  \texttt{grist@uns.ac.rs}
  \and
  Goran Sladić\\
  \texttt{sladicg@uns.ac.rs }
  \\Faculty of Technical Sciences,\\ University in Novi Sad, Serbia
}


\date{}
\maketitle


\begin{abstract}
The General Data Protection Regulation (GDPR) is a law regulation issued by the European Union. The GDPR was adopted on 14 April 2016 and became enforceable beginning 25 May 2018. This regulation aims to give individuals control over their personal data across the European Union and unify regulations within the EU to facilitate international business. This regulation is a mixed blessing, on the one hand, it has brought many benefits to individuals, but on the other hand, it has provided many challenges and obstacles to companies and organizations that control or process personal data. Despite that the GDPR provides a unified and centralized law regulation, there is no universal software solution for checking GDPR compliance. Therefore, to overcome some of the security challenges, companies pay expensive manual checking executed by lawyers or create their own solutions that do not apply to other organizations and industries. In this paper, we propose a solution in the form of a privacy meta-model and OCL validations which provides automatical checking for violations of the first fifty arts of the GDPR. Also, we are going to present usage of this meta-model on an example of a bank and plan of future development.
\end{abstract}

\section{Introduction}



\end{document}