\documentclass[11pt,english]{article}
\usepackage[utf8]{inputenc}
\usepackage[T1]{fontenc}
\usepackage{babel}

% correct bad hyphenation here
\hyphenation{op-tical net-works semi-conduc-tor}
\begin{document}
\title{Use of Model Driven Development to check for violations of the GDPR}
\author{
  Tijana Lalošević\\
  \texttt{tijana.vdn@gmail.com}
  \and
  Gordana Milosavljević\\
  \texttt{grist@uns.ac.rs}
  \and
  Goran Sladić\\
  \texttt{sladicg@uns.ac.rs }
  \\Faculty of Technical Sciences,\\ University in Novi Sad, Serbia
}


\date{}
\maketitle


\begin{abstract}
The General Data Protection Regulation (GDPR) is a law regulation issued by the European Union. The GDPR was adopted on 14 April 2016 and became enforceable beginning 25 May 2018. The main purpose of this regulation is to give individuals control over their personal data across the European Union and unify regulations within the EU to facilitate international business. This regulation is a mixed blessing, on the one hand, it has brought many benefits to individuals, but on the other hand, it has provided many challenges to companies and organizations that control or process personal data.
\end{abstract}


\section{Introduction}



\end{document}
